\documentclass{worksheet}
\usepackage{amsmath, amsthm}
\usepackage{lipsum}
\usepackage{minted}

\title{Example worksheet}
\author{Simon Cooksey}
\date{31 November 2022}

\newmintinline{ocaml}{}
\setminted{breaklines}
\newtheorem{lemma}{Lemma}

\begin{document}
\maketitle

This is an example worksheet with exercises denoted by big {\tt tcolorbox}s.

\section{Files}
There is a \mintinline{latex}{\filename} macro for typesetting filenames. This is useful for specifying submission structure.

\begin{minted}{latex}
Add your code to \filename{class1.ml}
\end{minted}

Add your code to \filename{class1.ml}

\section{Tasks}
\begin{minted}{latex}
\newmintinline{ocaml}{}
\newcommand{\theFile}{\filename{class1.ml}}

\begin{task*}[Some un-numbered task]
All tasks will be marked with a {\color{red!75!black} red} box!
These must all be completed for full marks.
\end{task*}

\begin{task}[Numbered task]
    \label{task:increment}
    In \theFile, define a new function named \ocamlinline/increment/ which takes a single argument and adds 1 to it.
    Your function must have type \ocamlinline/int -> int/.
\end{task}
\end{minted}

\begin{task*}[Some un-numbered task]
    \label{task:increment}
    All tasks will be marked with a {\color{red!75!black} red} box!
    These must all be completed for full marks.
\end{task*}

\newcommand{\theFile}{\filename{class1.ml}}
\begin{task}[Numbered task]
    In \theFile, define a new function named \ocamlinline/increment/ which takes a single argument and adds 1 to it.
    Your function must have type \ocamlinline/int -> int/.
\end{task}

\section{Solutions}
\begin{minted}{latex}
\begin{solution*}[to Task~\ref{task:increment}]
\mintinline{ocaml}{let increment x = x + 1}
\end{solution*}
\end{minted}

\begin{solution*}[to Task~\ref{task:increment}]
\mintinline{ocaml}{let increment x = x + 1}
\end{solution*}

\section{Examples}
\begin{minted}{latex}
\begin{example*}[Example proof using induction over the naturals]
\begin{lemma}[$3^n - 1$ is even]
    \label{lemma:power-of-three-even}
    $$\forall n. \exists x. 3^n - 1 = 2x$$
\end{lemma}
\begin{proof}[Proof of Lemma~\ref{lemma:power-of-three-even}]
\dots
\end{proof}
\end{example*}
\end{minted}

\begin{example*}[Example proof using induction over the naturals]
\begin{lemma}[$3^n - 1$ is even]
    \label{lemma:power-of-three-even}
    $$\forall n. \exists x. 3^n - 1 = 2x$$
\end{lemma}
\begin{proof}[Proof of Lemma~\ref{lemma:power-of-three-even}]
\dots
\end{proof}
\end{example*}

\section{Margin notes}
Margin notes are set up somewhat sensibly.

\begin{minted}{latex}
\marginpar{This text is nonsense, and this note isn't very helpful either.}
\lipsum[1]
\end{minted}

\marginpar{This text is nonsense, and this note isn't very helpful either.}
\lipsum[1]


\end{document}
